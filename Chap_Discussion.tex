% !TEX encoding = UTF-8 Unicode
\documentclass[BachelorPaper]{subfiles}
\acresetall
%Providecommands für Subfiles
    \providecommand{\citepic}[1]{(#1)}
    \providecommand{\citefig}[2]{(#1, S. #2)}
    \providecommand{\citefigm}[2]{(Modifiziert #1, S. #2)}

\begin{document}
\chapter{Discussion}
During the development of this application I faced situations in which I found myself lost in the complexity of modern web-centric software engineering. Especially the complex asynchronous nature of web communication and the problematic handling of information state and client authentication often blocked my advances but after finishing the first major step -- with the implementation of the first two main components the BOINSO Core Web Application and the BOINSO MCC Web Client -- I feel confident that this project can be a useful base for future development. \\

The majority of tools (and many more not used in this project) where new to me and it took some time to get accustomed to the different programming paradigms but I think that structured as they are they are a relatively future proof foundation for both the project and the way \acp{MCC} and \acp{GCC} are going to collaborate in the future. \\

The most problematic step is still the full implementation of the BOINSO GPredict Bridge package as it is a crucial part of the collaborative process. It still does not have enough functionality to be convenient for \ac{GCC} administrators who may be technical enthusiast but still favor the most comfortable solution. Also still not optimally solved is the upload of tracking data as this component is still too dependent on the output of the radio (or software defined radio solution). Right now there does not exist enough data about \acp{GCC} who use other solutions than FlDigi or similar products and the actual monitoring work-flow is still to obscure.\\

The next steps include the founding of partnerships with interested universities in Portugal, Denmark and Japan which have to be introduced to the general concept of the project and the further development. If this step can be taken and my old mistakes -- the code base is still by far not tested enough and the core model graph definitely needs a second opinion -- can be re-factored I see a good and prosperous future for the BOINSO project and maybe even new affiliated programs.\\

As a part of the biomedical engineering community I am especially interested in the future medical applications that can be build on our technology. As far as I can see -- if we succeed -- the efficiency of educational satellite projects should increase dramatically which leads to an increased amount of available satellite time and a bigger incentive for new satellite projects being launched. The additional satellite time can be used to consolidate programs and reserve time for testing transmission of medical data over satellites in \ac{LEO}. Colleagues have already proven that medical images -- like X-Rays of fractured bones -- can be transported via small radio devices in a timely manner. Regions without any other means of connection could upload and receive information using the nearest \ac{GCC} with actual internet connection as a proxy.\\

\end{document}