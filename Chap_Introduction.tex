% !TEX encoding = UTF-8 Unicode
\documentclass[BachelorPaper]{subfiles}
\acresetall
%Providecommands für Subfiles
    \providecommand{\citepic}[1]{(#1)}
    \providecommand{\citefig}[2]{(#1, S. #2)}
    \providecommand{\citefigm}[2]{(Modifiziert #1, S. #2)}

\begin{document}
\chapter{Introduction}
The use of pico and micro satellites in the context of educational aerospace programs has introduced opportunities for both private enthusiasts and students in small scale cost efficient satellite endeavors. Mostly lacking central oversight and costly infrastructure around the globe the efficiency of those programs is depending on the collaborative efforts of a growing community sharing both hardware, computational infrastructure and time. Due to the characteristics of orbiters deployed in a \ac{LEO} a stationary observer  can establish a direct line of sight only up to twelve times a day -- as stated in \cite{kief_genso_2011} -- which can be maintained for about fifteen minutes per pass.

The challenges posed by the difficult monitoring of small satellites orbiting in \ac{LEO} and possible ways to overcome them were outlined in \cite{beyerle_boinso_2014}. This bachelor paper focuses on the implementation of an accessible and easily modifiable solution to connect \acp{MCC} and give \acp{GCC} the opportunity to deliver viable contributions to the monitoring process.

\end{document}
