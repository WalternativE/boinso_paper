After the start of the CubeSat program in 1999 pico and micro satellites began to make an appearance in educational space flight. Today spacecrafts in this size category are used by research institutions to train engineers and collect large amounts of data. As bigger satellites in higher orbits are more expensive even commercial satellite missions start to venture in this area. Due to the lower costs and the short life period of those devices they orbit in an altitude of 120--1200~km. Objects in these  heights do not have an geostationary orbit which leads to difficulties while establishing regular radio contact. A mission control center tracking a satellite has about 15 minutes per pass which occurs up to twelve times per day in optimal conditions for radio communication. This is insufficient for most endeavors. \\

Currently there are hundreds of educational and amateur radio stations equipped to establish satellite communication. The operators of those stations track satellites and transmit down-link data to mission control independently, and are mostly uncoordinated. The original mission control center and ground stations configured to track a single satellite have to rest in an idle state for the majority of the time. Projects such as Global Educational Network for Satellite Operations tackle the issue of the poor communication coverage and the ineffective use of ground station resources. However it has yet to offer a viable option for projects which require an individual communication network approach without connecting to a centralized system. \\

In this paper it is the goal to tackle this issue by implementing a mission control and ground station network using a set of modern web application tools and architectural paradigms. \\