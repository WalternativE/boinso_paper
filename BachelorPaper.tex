%
% FH Technikum Wien
% !TEX encoding = UTF-8 Unicode
%
% Erstellung von Master- und Bachelorarbeiten an der FH Technikum Wien mit Hilfe von LaTeX und der Klasse TWBOOK
%
% Um ein eigenes Dokument zu erstellen, müssen Sie folgendes ergänzen:
% 1) Mit \documentclass[..] einstellen: Master- oder Bachelorarbeit, Studiengang und Sprache
% 2) Mit \newcommand{\FHTWCitationType}.. Zitierstandard festlegen (wird in der Regel vom Studiengang vorgegeben - bitte erfragen)
% 3) Deckblatt, Kurzfassung, etc. ausfüllen
% 4) und die Arbeit schreiben (die verwendeten Literaturquellen in Literatur.bib eintragen)
%
% Getestet mit TeXstudio mit Zeichenkodierung ISO-8859-1 (=ansinew/latin1) und MikTex unter Windows
% Zu beachten ist, dass die Kodierung der Datei mit der Kodierung des paketes inputenc zusammen passt!
% Die Kodierung der Datei twbook.cls MUSS ANSI betragen!
% Bei der Verwendung von UTF8 muss nicht nur die Kodierung des Dokuments auf UTF8 gestellt sein, sondern auch die des BibTex-Files!
%
% Bugreports und Feedback bitte per E-Mail an latex@technikum-wien.at
%
% Versionen
% *) V0.7: 9.1.2015, RO: Modeline angepasst und verschoben
% *) V0.6: 10.10.2014, RO: Weitere Anpassung an die UK
% *) V0.5: 8.8.2014, WK: Literaturquellen überarbeitet und angepasst
% *) V0.4: 4.8.2014, WK: Initalversion in SVN eingespielt
%
\documentclass[Bachelor,BBE,english]{twbook}
\usepackage[utf8]{inputenc}
\usepackage[T1]{fontenc}

%
% Bitte in der folgenden Zeile den Zitierstandard festlegen
\newcommand{\FHTWCitationType}{IEEE} % IEEE oder HARVARD möglich - wenn Sie zwischen IEEE und HARVARD wechseln, bitte die temorären Dateien (aux, bbl, ...) löschen
%
\ifthenelse{\equal{\FHTWCitationType}{HARVARD}}{\usepackage{harvard}}{\usepackage{bibgerm}}

% Definition Code-Listings Formatierung:
\usepackage[final]{listings}
\lstset{captionpos=b, numberbychapter=false,caption=\lstname,frame=single, numbers=left, stepnumber=1, numbersep=2pt, xleftmargin=15pt, framexleftmargin=15pt, numberstyle=\tiny, tabsize=3, columns=fixed, basicstyle={\fontfamily{pcr}\selectfont\footnotesize}, keywordstyle=\bfseries, commentstyle={\color[gray]{0.33}\itshape}, stringstyle=\color[gray]{0.25}, breaklines, breakatwhitespace, breakautoindent}
\lstloadlanguages{[ANSI]C, C++, [gnu]make, gnuplot, Matlab}

%Formatieren des Quellcodeverzeichnisses
\makeatletter
% Setzen der Bezeichnungen für das Quellcodeverzeichnis/Abkürzungsverzeichnis in Abhängigkeit von der eingestellten Sprache
\providecommand\listacroname{}
\@ifclasswith{twbook}{english}
{%
    \renewcommand\lstlistingname{Code}
    \renewcommand\lstlistlistingname{List of Code}
    \renewcommand\listacroname{List of Abbreviations}
}{%
    \renewcommand\lstlistingname{Quellcode}
    \renewcommand\lstlistlistingname{Quellcodeverzeichnis}
    \renewcommand\listacroname{Abkürzungsverzeichnis}
}
% Wenn die Option listof=entryprefix gewählt wurde, Definition des Entyprefixes für das Quellcodeverzeichnis. Definition des Macros listoflolentryname analog zu listoflofentryname und listoflotentryname der KOMA-Klasse
\@ifclasswith{scrbook}{listof=entryprefix}
{%
    \newcommand\listoflolentryname\lstlistingname
}{%
}
\makeatother
\newcommand{\listofcode}{\phantomsection\lstlistoflistings}

%SI-Units
\usepackage{siunitx}
%\sisetup{range-phrase={ -- }, list-final-separator={ und }, output-decimal-marker={,}, group-minimum-digits=3, per-mode=symbol, binary-units}
\sisetup{range-phrase={ -- }, list-final-separator={ und }, output-decimal-marker={,}, group-minimum-digits=3, binary-units}
%Aufteilen in mehrere Dateien
\usepackage{subfiles}

% for including other pdf files in appendix
\usepackage{pdfpages}

% my own listing definitions ---------------------------------------------------------------

\definecolor{lightgray}{rgb}{.9,.9,.9}
\definecolor{darkgray}{rgb}{.4,.4,.4}
\definecolor{purple}{rgb}{0.65, 0.12, 0.82}

\lstdefinelanguage{JavaScript}{
  keywords={typeof, new, true, false, catch, function, return, null, catch, switch, var, if, in, while, do, else, case, break},
  keywordstyle=\color{blue}\bfseries,
  ndkeywords={class, export, boolean, throw, implements, import, this},
  ndkeywordstyle=\color{darkgray}\bfseries,
  identifierstyle=\color{black},
  sensitive=false,
  comment=[l]{//},
  morecomment=[s]{/*}{*/},
  commentstyle=\color{purple}\ttfamily,
  stringstyle=\color{red}\ttfamily,
  morestring=[b]',
  morestring=[b]"
}

\lstset{
   language=JavaScript,
   backgroundcolor=\color{lightgray},
   extendedchars=true,
   basicstyle=\footnotesize\ttfamily,
   showstringspaces=false,
   showspaces=false,
   numbers=left,
   numberstyle=\footnotesize,
   numbersep=9pt,
   tabsize=2,
   breaklines=true,
   showtabs=false,
   captionpos=b
}

\newcommand\YAMLcolonstyle{\color{red}\mdseries}
\newcommand\YAMLkeystyle{\color{black}\bfseries}
\newcommand\YAMLvaluestyle{\color{blue}\mdseries}

\makeatletter

% here is a macro expanding to the name of the language
% (handy if you decide to change it further down the road)
\newcommand\language@yaml{yaml}

\expandafter\expandafter\expandafter\lstdefinelanguage
\expandafter{\language@yaml}
{
  keywords={true,false,null,y,n},
  keywordstyle=\color{darkgray}\bfseries,
  basicstyle=\YAMLkeystyle,                                 % assuming a key comes first
  sensitive=false,
  comment=[l]{\#},
  morecomment=[s]{/*}{*/},
  commentstyle=\color{purple}\ttfamily,
  stringstyle=\YAMLvaluestyle\ttfamily,
  moredelim=[l][\color{orange}]{\&},
  moredelim=[l][\color{magenta}]{*},
  moredelim=**[il][\YAMLcolonstyle{:}\YAMLvaluestyle]{:},   % switch to value style at :
  morestring=[b]',
  morestring=[b]",
  literate =    {---}{{\ProcessThreeDashes}}3
                {>}{{\textcolor{red}\textgreater}}1     
                {|}{{\textcolor{red}\textbar}}1 
                {\ -\ }{{\mdseries\ -\ }}3,
}

% switch to key style at EOL
\lst@AddToHook{EveryLine}{\ifx\lst@language\language@yaml\YAMLkeystyle\fi}
\makeatother

\newcommand\ProcessThreeDashes{\llap{\color{cyan}\mdseries-{-}-}}

% end of listing definitions -------------------------------------------------------------

%
% Einträge für Deckblatt, Kurzfassung, etc.
%
\title{BOINSO -- Open source development of a distributed satellite monitoring network and its implications on telemedicine applications of the future}
\author{Gregor Beyerle}
\studentnumber{1210227035}
\supervisor{FH-Prof. Dipl. Ing. Dr. Lars Mehnen}
\place{Vienna}
\kurzfassung{Nach dem Start des CubeSat Programmes 1999 begannen Micro- und Pico-Satelliten  ihren Platz in Raumfahrtprogrammen einzunehmen. Heutzutage werden Flugkörper dieser Größenordnung von Forschungseinrichtungen verwendet um Ingenieure auszubilden und selbst kostengünstig Forschung zu betreiben. Selbst kommerzielle Anbieter entwickeln derzeit in diesem Bereich, da größere und schwerere Satelliten, welche höher gelegene Orbits einnehmen, weitaus teurer sind. Auf Grund von Kostenersparnissen und der kurzen Lebensdauer von Kleinsatelliten werden diese in Umlaufbahnen gebracht die nur 120--1200~km über dem Meeresspiegel liegen. Objekte in solchen Höhen besitzen keinen geostationären Orbit, was dazu führt, dass von einer Bodenstation kein durchgängiger Funkkontakt hergestellt werden kann. Ein Kontrollzentrum hat daher nur zu etwa zwölf Terminen pro Tag Kommunikationsfenster von ungefähr fünfzehn Minuten. Dieser Umstand ist nur schwer hinnehmbar. \\

Weltweit gibt es hunderte Einrichtungen, welche in der Lage wären Verbindung zu erdnahen Satelliten aufzunehmen. Betreiber dieser Stationen arbeiten schon jetzt daran Satelliten manuell zu verfolgen, Daten zu empfangen und diese den zugehörigen Kontrollzentren weiterzuleiten. Dieser Prozess findet für gewöhnlich ohne Koordination der Kontrollzentren statt und erfolgt meist bruchstückhaft und unzuverlässig. Stationen, welche auf einen Satelliten eingestellt wurden verbleiben die meiste Zeit in Warteposition. Projekte wie das "Global Educational Network for Satellite Operations" versuchen die schwache Funknetzabdeckung sowie die Nutzungseffizienz der Sende- und Empfangsstationen zu verbessern. Für Missionen, die sich nicht von einem stark zentralisierten Netzwerk abhängig sein wollen gibt es allerdings noch keine veritablen Alternativen. \\

Diese Arbeit beschäftigt sich mit der Implementation eines verteilten Netzwerkes von Bodenstationen unter Anwendung moderner Web-Technologien und Architekturparadigmen.}
\schlagworte{BOINSO, Open Source, CubeSat}
\outline{After the start of the CubeSat program in 1999 pico and micro satellites began to make an appearance in educational space flight. Today spacecrafts in this size category are used by research institutions to train engineers and collect large amounts of data. As bigger satellites in higher orbits are more expensive even commercial satellite missions start to venture in this area. Due to the lower costs and the short life period of those devices they orbit in an altitude of 120--1200~km. Objects in these  heights do not have an geostationary orbit which leads to difficulties while establishing regular radio contact. A mission control center tracking a satellite has about 15 minutes per pass which occurs up to twelve times per day in optimal conditions for radio communication. This is insufficient for most endeavors. \\

Currently there are hundreds of educational and amateur radio stations equipped to establish satellite communication. The operators of those stations track satellites and transmit down-link data to mission control independently, and are mostly uncoordinated. The original mission control center and ground stations configured to track a single satellite have to rest in an idle state for the majority of the time. Projects such as Global Educational Network for Satellite Operations tackle the issue of the poor communication coverage and the ineffective use of ground station resources. However it has yet to offer a viable option for projects which require an individual communication network approach without connecting to a centralized system. \\

In this paper it is the goal to tackle this issue by implementing a mission control and ground station network using a set of modern web application tools and architectural paradigms. \\}
\keywords{BOINSO, Open Source, CubeSat}
\acknowledgements{I want to thank my colleague Marcel Cimander for working with me on this project for the last year and his great work on GPredict which will hopefully soon be a part of the official release cycle. I would also like to thank Alexandru Csete for his wonderful work on GPredict, the fact that he also still maintains the Win32 build of the software and for his kind consideration.\\

I also want to thank Mr Dipl.-Ing. Mag. Christian Siehs for "pushing" me into web development as otherwise I would most likely not have been able to pursue this project the way I did. Additionally I would like thank my girlfriend who had to cope with me for the time I was working on this project and proved an extreme resilience to my grumpiness when facing a major bug.\\

Finally I would like to thank Mr FH-Prof. Dipl. Ing. Dr. Mehnen, who approached and trusted us with this very demanding but also very rewarding project. I would never have dreamed of actually working on a project of such a scale within my undergraduate program.}
\begin{document}

%Festlegungen für den HARVARD-Zitierstandard
\ifthenelse{\equal{\FHTWCitationType}{HARVARD}}{
\bibliographystyle{Harvard_FHTW_MR}%Zitierstandard FH Technikum Wien, Studiengang Mechatronik/Robotik, Version 1.2e
\citationstyle{dcu}%Correct citation-style (Harvardand, ";" between citations, "," between author and year)
\citationmode{abbr}%use "et al." with first citation
\iflanguage{ngerman}{
    %Deutsch Neue Rechtschreibung
    \newcommand{\citepic}[1]{(Quelle: \protect\cite{#1})}%Zitat: Bild
    \newcommand{\citefig}[2]{(Quelle: \protect\cite{#1}, S. #2)}%Zitat: Bild aus Dokument
    \newcommand{\citefigm}[2]{(Quelle: modifiziert "ubernommen aus \protect\cite{#1}, S. #2)}%Zitat: modifiziertes Bild aus Dokument
    \newcommand{\citep}{\citeasnoun}%In-Line Zitiat entweder mit \citep{} oder \citeasnoun{}
    \newcommand{\acessedthrough}{Verf{\"u}gbar unter:}%Für URL-Angabe
    \newcommand{\acessedthroughp}{Verf{\"u}gbar bei:}%Für URL-Angabe (Geschützte Datenbank, Zugriff durch FH)
    \newcommand{\acessedat}{Zugang am}%Für URL-Datum-Angabe
    \newcommand{\singlepage}{S.}%Für Seitenangabe (einzelne Seite)
    \newcommand{\multiplepages}{S.}%Für Seitenangabe (mehrere Seiten)
    \newcommand{\chapternr}{K.}%Für Kapitelangabe
    \renewcommand{\harvardand}{\&}%Harvardand in Zitaten
    \newcommand{\abstractonly}{ausschließlich Abstract}
    \newcommand{\edition}{. Auflage}%Angabe der Auflage
}{
\iflanguage{german}{
    %Deutsch
    \newcommand{\citepic}[1]{(Quelle: \protect\cite{#1})}%Zitat: Bild
    \newcommand{\citefig}[2]{(Quelle: \protect\cite{#1}, S. #2)}%Zitat: Bild aus Dokument
    \newcommand{\citefigm}[2]{(Quelle: modifiziert "ubernommen aus \protect\cite{#1}, S. #2)}%Zitat: modifiziertes Bild aus Dokument
    \newcommand{\citep}{\citeasnoun}%In-Line Zitiat entweder mit \citep{} oder \citeasnoun{}
    \newcommand{\acessedthrough}{Verf{\"u}gbar unter:}%Für URL-Angabe
    \newcommand{\acessedthroughp}{Verf{\"u}gbar bei:}%Für URL-Angabe (Geschützte Datenbank, Zugriff durch FH)
    \newcommand{\acessedat}{Zugang am}%Für URL-Datum-Angabe
    \newcommand{\singlepage}{S.}%Für Seitenangabe (einzelne Seite)
    \newcommand{\multiplepages}{S.}%Für Seitenangabe (mehrere Seiten)
    \newcommand{\chapternr}{K.}%Für Kapitelangabe
    \renewcommand{\harvardand}{\&}%Harvardand in Zitaten
    \newcommand{\abstractonly}{ausschließlich Abstract}
    \newcommand{\edition}{. Auflage}%Angabe der Auflage
}{
    %Englisch
    \newcommand{\citepic}[1]{(Source: \protect\cite{#1})}%Zitat: Bild
    \newcommand{\citefig}[2]{(Source: \protect\cite{#1}, p. #2)}%Zitat: Bild aus Dokument
    \newcommand{\citefigm}[2]{(Source: taken with modification from \protect\cite{#1}, p. #2)}%Zitat: modifiziertes Bild aus Dokument
    \newcommand{\citep}{\citeasnoun}%In-Line Zitiat entweder mit \citep{} oder \citeasnoun{}
    \newcommand{\acessedthrough}{Available at:}%Für URL-Angabe
    \newcommand{\acessedthroughp}{Available through:}%Für URL-Angabe (Geschützte Datenbank, Zugriff durch FH)
    \newcommand{\acessedat}{Accessed}%Für URL-Datum-Angabe
    \newcommand{\singlepage}{p.}%Für Seitenangabe (einzelne Seite)
    \newcommand{\multiplepages}{pp.}%Für Seitenangabe (mehrere Seiten)
    \newcommand{\chapternr}{Ch.}%Für Kapitelangabe
    \renewcommand{\harvardand}{\&}%Harvardand in Zitaten
    \newcommand{\abstractonly}{Abstract only}
    \newcommand{\edition}{~edition}%Edition -> note, that you have to write "edition = {2nd},"!
}}}

\maketitle

%
% .. und hier beginnt die eigentliche Arbeit. Viel Erfolg beim Verfassen!
%

\subfile{Chap_Introduction}
\clearpage
\subfile{Chap_Materials}
\clearpage
\subfile{Chap_Methods}
\clearpage
\subfile{Chap_Results}
\clearpage
\subfile{Chap_Discussion}

%
% Hier beginnen die Verzeichnisse.
%
\clearpage
\ifthenelse{\equal{\FHTWCitationType}{HARVARD}}{}{\bibliographystyle{IEEEtran}}
\bibliography{Literatur}
\clearpage

% Das Abbildungsverzeichnis
\listoffigures
\clearpage

% Das Tabellenverzeichnis
\listoftables
\clearpage

% Das Quellcodeverzeichnis
\listofcode
\clearpage

\phantomsection
\addcontentsline{toc}{chapter}{\listacroname}
\chapter*{\listacroname}
\begin{acronym}[XXXXX]
   \IfFileExists{Acro_sorted}{\acro{API}{Application Programming Interface}
\acro{BOINC}{Berkeley Open Infrastructure for Network Computing}
\acro{BSD}{Berkeley Software Distribution}
\acro{CGI}{Common Gateway Interface}
\acro{CORS}{Cross-Origin Resource Sharing}
\acro{COTS}{Commercial off-the-shelf}
\acro{CPU}[CPUs]{Central Processing Units}
\acro{CSA}{Canadian Space Agency}
\acro{CSS}[CSS]{Cascading Style Sheets}
\acro{ESA}{European Space Agency}
\acro{FSS}{Free Software Foundation}
\acro{GCC}[GCCs]{Ground Control Centers}
\acro{GENSO}{Global Educational Network for Satellite Operations}
\acro{GEO}{Geostationary Orbit}
\acro{GPL}{GNU General Public License}
\acro{GPLv2}{GNU General Public Licence Version 2}
\acro{GPU}[GPUs]{Graphics Processing Units}
\acro{HATEOAS}{Hypermedia as the Engine of Application State}
\acro{HTML}{Hypertext Markup Language}
\acro{HTTP}{Hypertext Transfer Protocol}
\acro{HamLib}{Ham Radio Control Libraries}
\acro{IIS}{Microsoft Internet Information Services}
\acro{ISEB}{International Space Education Board}
\acro{JAXA}{Japan Aerospace Exploration Agency}
\acro{JSON}{JavaScript Object Notation}
\acro{LEO}{Low Earth Orbit}
\acro{LTS}{Long Term Support}
\acro{MCC}[MCCs]{Mission Control Centers}
\acro{MIT}{Massachusetts Institute of Technology}
\acro{MPL-2.0}{Mozilla Public License Version 2.0}
\acro{MVC}{Model View Controller}
\acro{NASA}{National Aeronautics and Space Administration}
\acro{NORAD}{North American Aerospace Defense Command}
\acro{ORM}{Object Relational Mapping}
\acro{PRC}{Public Resource Computing}
\acro{PyPI}{Python Package Index}
\acro{REST}{Representational State Transfer}
\acro{SSL}{Secure Socket Layer}
\acro{SVN}{Subversion}
\acro{TLE}{Two Line Elements}
\acro{TLS}{Transport Level Security}
\acro{TNC}{Terminal Node Controller}
\acro{URL}[URLs]{Uniform Resource Locators}
\acro{WSGI}{Web Server Gateway Interface}
\acro{git-scm}{Git Source Code Mirror}
\acro{npm}{Node Package Manager}
}{\acro{API}{Application Programming Interface}
\acro{GENSO}{Global Educational Network for Satellite Operations}
\acro{BOINC}{Berkeley Open Infrastructure for Network Computing}
\acro{LEO}{Low Earth Orbit}
\acro{GEO}{Geostationary Orbit}
\acro{GPL}{GNU General Public License}
\acro{MCC}{Mission Control Center}
\acrodefplural{MCC}[MCCs]{Mission Control Centers}
\acro{GCC}{Ground Control Center}
\acrodefplural{GCC}[GCCs]{Ground Control Centers}
\acro{COTS}{Commercial off-the-shelf}
\acro{ISEB}{International Space Education Board}
\acro{CSA}{Canadian Space Agency}
\acro{ESA}{European Space Agency}
\acro{JAXA}{Japan Aerospace Exploration Agency}
\acro{NASA}{National Aeronautics and Space Administration}
\acro{CPU}{Central Processing Unit}
\acrodefplural{CPU}[CPUs]{Central Processing Units}
\acro{GPU}{Graphics Processing Unit}
\acrodefplural{GPU}[GPUs]{Graphics Processing Units}
\acro{PRC}{Public Resource Computing}
\acro{TNC}{Terminal Node Controller}
\acro{LTS}{Long Term Support}
\acro{URL}{Uniform Resource Locator}
\acro{CGI}{Common Gateway Interface}
\acro{MPL-2.0}{Mozilla Public License Version 2.0}
\acro{NORAD}{North American Aerospace Defense Command}
\acro{HamLib}{Ham Radio Control Libraries}
\acro{TLE}{Two Line Elements}
\acro{git-scm}{Git Source Code Mirror}
\acro{SVN}{Subversion}
\acro{GPLv2}{GNU General Public Licence Version 2}
\acro{REST}{Representational State Transfer}
\acro{PyPI}{Python Package Index}}
\end{acronym}
\clearpage


%
% Hier beginnt der Anhang.
%
\appendix
\subfile{BOINSO_doc_appendix}

\end{document}
