Nach dem Start des CubeSat Programmes 1999 begannen Micro- und Pico-Satelliten  ihren Platz in Raumfahrtprogrammen einzunehmen. Heutzutage werden Flugkörper dieser Größenordnung von Forschungseinrichtungen verwendet um Ingenieure auszubilden und selbst kostengünstig Forschung zu betreiben. Selbst kommerzielle Anbieter entwickeln derzeit in diesem Bereich, da größere und schwerere Satelliten, welche höher gelegene Orbits einnehmen, weitaus teurer sind. Auf Grund von Kostenersparnissen und der kurzen Lebensdauer von Kleinsatelliten werden diese in Umlaufbahnen gebracht die nur 120--1200~km über dem Meeresspiegel liegen. Objekte in solchen Höhen besitzen keinen geostationären Orbit, was dazu führt, dass von einer Bodenstation kein durchgängiger Funkkontakt hergestellt werden kann. Ein Kontrollzentrum hat daher nur zu etwa zwölf Terminen pro Tag Kommunikationsfenster von ungefähr fünfzehn Minuten. Dieser Umstand ist nur schwer hinnehmbar. \\

Weltweit gibt es hunderte Einrichtungen, welche in der Lage wären Verbindung zu erdnahen Satelliten aufzunehmen. Betreiber dieser Stationen arbeiten schon jetzt daran Satelliten manuell zu verfolgen, Daten zu empfangen und diese den zugehörigen Kontrollzentren weiterzuleiten. Dieser Prozess findet für gewöhnlich ohne Koordination der Kontrollzentren statt und erfolgt meist bruchstückhaft und unzuverlässig. Stationen, welche auf einen Satelliten eingestellt wurden verbleiben die meiste Zeit in Warteposition. Projekte wie das "Global Educational Network for Satellite Operations" versuchen die schwache Funknetzabdeckung sowie die Nutzungseffizienz der Sende- und Empfangsstationen zu verbessern. Für Missionen, die sich nicht von einem stark zentralisierten Netzwerk abhängig sein wollen gibt es allerdings noch keine veritablen Alternativen. \\

Diese Arbeit beschäftigt sich mit der Implementation eines verteilten Netzwerkes von Bodenstationen unter Anwendung moderner Web-Technologien und Architekturparadigmen.