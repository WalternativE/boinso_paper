% !TEX encoding = UTF-8 Unicode
\documentclass[BachelorPaper]{subfiles}
\acresetall
%Providecommands für Subfiles
    \providecommand{\citepic}[1]{(#1)}
    \providecommand{\citefig}[2]{(#1, S. #2)}
    \providecommand{\citefigm}[2]{(Modifiziert #1, S. #2)}

\begin{document}
\chapter{Material and Methods}
The following sections list the different tools used while working on the components of the BOINSO network applications.

\section{General}
This section includes tools which allow a group of developers to work on complex projects in a decentralized manner. As this project is intended to be used in an academic context and monetary resources are scarce in this field tools that are either open source or free of charge for educational programs.

\subsection{GIT-SCM}
The \ac{git-scm} is a "[...] fast, scalable, distributed revison control system with a [...] rich command set that provides both high-level operations and full access to internals" as declared in \cite{git_scm}. It follows the same "branch -> develop -> merge" work-flow as \ac{SVN} and is freely distributed under the \ac{GPLv2}. It was originally developed by Linus Torvalds to offer Linux Kernel developers a free way to collaboratively work on a distributed code-base efficiently.

\subsection{GitHub}
GitHub is a provider of cloud hosted \ac{git-scm} repositories. As GitHub was originally introduced by members of the open source community it still maintains a very generous relationship to open source contributors. GitHub users who publish their work to public repositories may use virtually all services bound to the GitHub infrastructure free of charge with only minimal limitations. GitHub also offers premium enterprise accounts which include a certain amount of private repositories and premium services if they are needed.

\subsection{Travis-CI}
Travis-CI is a web hosted continuous integration server. Continuous integration is the automated process of building and testing a project with every introduction or modification of a software component. Its goal is to assure and improve the quality of the project while alerting the development team if a change would lead to a breaking application.\\

In contrast of other continuous integration solutions like Jenkins -- an open source continuous integration server implemented in the Java programming language -- the configuration of a Travis-CI process is done by adding a simple configuration file to root of your \ac{git-scm} repository as seen in listing \ref{lst:travis.yml}.\\

\lstinputlisting[language=yaml, caption={BOINSO travis config file}, label=lst:travis.yml]{listings/.travis.yml}

\subsection{Vagrant}
Vagrant is a tool which is used to virtualize development environments. It offers a command line interface to download, start up, pause, halt and provision images of virtual machines which come configured with all the dependencies needed to develop, run and test an application. Base images can be built to closely model a production system as closely as possible being configured by a trained system administrator masking the complexity of this system from developers and designers. Depending on the base image type Vagrant users have to have access to a certain virtualization provider -- also known as hypervisor.\\

The initialization process of a Vagrant environment depends on the presence of a vagrant configuration file -- simply known as the Vagrantfile. Listing \ref{lst:Vagrantfile} depicts the Vagrantfile of the BOINSO Core Web Application.\\

\lstinputlisting[language=Ruby, caption={BOINSO Core Web Application Vagrantfile including expressions to set the virtual base box, a simple shell provisioner executing a bash script which installs variable project dependencies, and the automated port forwarding from the virtual box to the host development machine}, label=lst:Vagrantfile]{listings/Vagrantfile.rb}

Provisioning can be done by DevOps tools like Puppet or Chef or any other program used in this context. As the scope and the resources of this project are limited a simple shell provisioner was used. An example for a provisioning script can be seen in listing \ref{lst:bootstrap_prov}.\\

\lstinputlisting[language=bash, caption={Vagrant shell provisioning bash script}, label=lst:bootstrap_prov]{listings/bootstrap.sh}

\section{BOINSO Core Web Application}
This section focuses on the tools and frameworks used to implement the core of the BOINSO Core Web Application.

\subsection{Pip}
In a modern Python environment pip is used to manage Python modules. Pip itself is a python module which accesses the \ac{PyPI} downloads a module and its dependencies, starts the compilation process for Python extensions and either adds the module to the global Python interpreters Python path or the path of a local virtual Python environment. Besides installing 


\end{document}
